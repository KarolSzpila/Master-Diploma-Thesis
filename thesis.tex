%Przykładowy plik ułatwiający złożenie projektu dyplomowego inżynierskiego.
%UWAGA: Generowany napis na stronie tytułowej o treści PROJEKT DYPLOMOWY INŻYNIERSKI został zaproponowany przeze mnie i nie jest, póki co, potwierdzony przez władze wydziału. Przed ostatecznym oddaniem tak złożonej pracy należy upewnić się jaka powinna być treść tego napisu. W momencie gdy uzyskam informację na temat treści tego napisu, dokonam niezbędnych zmian w źródłach.

\documentclass[en,printmode]{mgr}
%opcje klasy dokumentu mgr.cls zostały opisane w dołączonej instrukcji

%poniżej deklaracje użycia pakietów, usunąć to co jest niepotrzebne
%\usepackage{polski} %przydatne podczas składania dokumentów w j. polskim
\usepackage[utf8]{inputenc}
\usepackage[T1]{fontenc} %poprawne składanie polskich czcionek
%pakiety do grafiki
\usepackage{graphicx}
\usepackage{subcaption}
\usepackage{psfrag}

%pakiety dodające dużo dodatkowych poleceń matematycznych
\usepackage{amsmath}
\usepackage{amsfonts}

%pakiety wspomagające i poprawiające składanie tabel
\usepackage{supertabular}
\usepackage{array}
\usepackage{tabularx}
\usepackage{hhline}
\usepackage{multirow}
\usepackage{indentfirst}
\usepackage{enumitem}

\usepackage{breqn}

\newcommand{\floor}[1]{\left\lfloor #1 \right\rfloor}
\usepackage[justification=centering]{caption}

%pakiet wypisujący na marginesie etykiety równań i rysunków zdefiniowanych przez \label{}, chcąc wygenerować finalną wersję dokumentu wystarczy usunąć poniższą linię
%\usepackage{showlabels}


%definicje własnych poleceń
\newcommand{\R}{I\!\!R} %symbol liczb rzeczywistych, działa tylko w trybie matematycznym
\newtheorem{theorem}{Twierdzenie}[section] %nowe otoczenie do składania twierdzeń

%dane do złożenia strony tytułowej
\title{System lokalizacji samolotów z wykorzystaniem ADS-B}
\engtitle{Airplane tracking system using ADS-B}
\author{Karol Szpila}
\supervisor{\vfil Ph.D., D.Sc. Grzegorz Budzyń\\
\\ Katedra Teorii Pola, Układów Elektronicznych i Optoelektroniki}
%\guardian{dr hab. inż. Imię Nazwisko Prof. PWr, I-6} %nie używać jeśli opiekun jest tą samą osobą co prowadzący pracę

%\date{2008} %standardowo u dołu strony tytułowej umieszczany jest bieżący rok, to polecenie pozwala wstawić dowolny rok

%poniżej jest lista kierunków i specjalności na wydziale elektroniki, należy wybrać właściwe lub dopisać jeśli nie ma odpowiednich
\field{Elektronika (EKA)}
\specialisation{Advanced Applied Electronics(AAE)}
%\specialisation{Robotyka (ARR)}
%\specialisation{Komputerowe sieci sterowania (ARK)}
%\specialisation{Systemy informatyczne w automatyce (ASI)}
%\specialisation{Komputerowe systemy zarządzania \\procesami produkcyjnymi (ARS)}
%\field{Elektronika i telekomunikacja (EIT)}
%\specialisation{Akustyka (ETA)}
%\specialisation{Aparatura elektroniczna (EAE)}
%\specialisation{Elektroniczne i komputerowe \\systemy automatyki (ESA)}
%\specialisation{Zastosowania inżynierii komputerowej \\w technice (EZI)}
%\specialisation{Inżynieria dźwięku (EID)}
%\specialisation{Elektronika stosowana \\i optokomunikacja (TEO)}
%\specialisation{Telekomunikacyjne sieci szerokopasmowe (TSS)}
%\specialisation{Teleinformatyczne sieci mobilne (TSM)}
%\specialisation{Sygnały w telekomunikacji cyfrowej (TSC)}
%\specialisation{Teleinformatyczne systemy rozsiewcze (TSR)}
%\field{Informatyka (INF)}
%\specialisation{Systemy informatyki w medycynie \\i technice (IMT)}
%\specialisation{Inżynieria systemów informatycznych (INS)}
%\specialisation{Inżynieria internetowa (INT)}
%\specialisation{Systemy i sieci komputerowe (ISK)}
%\field{Teleinformatyka (TIN)}
%\specialisation{Teleinformatyka (TIN)}

%tutaj zaczyna się właściwa treść dokumentu
\begin{document}
%\bibliographystyle{plabbrv} %tylko gdy używamy BibTeXa, ustawia polski styl bibliografii

\maketitle %polecenie generujące stronę tytułową
%\dedication{6cm}{To jest przykładowa treść opcjonalnej dedykacji, należy ją zmienić lub usunąć w całości polecenie \texttt{$\backslash$dedication}}

\tableofcontents %spis treści

%poniżej znajduje się przykładowa treść dalszej części dokumentu, zainteresowanych zachęcam do rozszyfrowania frazy "Lorem ipsum" :)
\let\cleardoublepage\clearpage %usuwa puste strony pomiaedzy rozdziałami

\chapter{Introduction}
	\section{Purpose and aim}
	\section{IQ signal model}
	\section{Theoretical operation of the mixer}
	\section{Software Defined Radio}
	\section{IQ imbalance models}
	
\chapter{Hardware and tools}
	\section{Zynq and Xilinx tools}
	\section{Adalm Pluto and AD tools}
	\section{Simulation environment}
	
\chapter{Algorithms}
	\section{DC offset correction}
		Moving average filter and Gaussian filter
	\section{Magnitude correction}
	\section{Phase correction}
		Blind phase correction algorithm

\chapter{Simulations}
Chapter with all algorithms simulation in Matlab.
	\section{Single tone signal}
	\section{Multitone signal}
	\section{QAM modulation}
	
\chapter{Measurements}
Chapter with all algorithms implemented in Zynq PL.
	\section{Single tone signal}
	\section{Multitone signal}
	\section{QAM modulation}


\chapter{ Conclusions}
zxZXasdasd

\addcontentsline{toc}{chapter}{Bibliography} %utworzenie w spisie treści pozycji Bibliografia
\bibliography{bibliografia} % wstawia bibliografię korzystając z pliku bibliografia.bib - dotyczy BibTeXa, jeżeli nie korzystamy z BibTeXa należy użyć otoczenia thebibliography

\begin{thebibliography}{9}
\bibitem{highSpeedDesign} 
Stephen H. Hal, Garrett W. Hall, and James A. McCall. 
\textit{High-Speed Digital System Design - A Handbook of Interconnects Theory and Design Practices}.
New York, Chichester, Weinheim, Brisbane, Singapore, Toronto, 2000.


\end{thebibliography}
%opcjonalnie może się tu pojawić spis rysunków i tabel
% \listoffigures
% \listoftables
\end{document}
